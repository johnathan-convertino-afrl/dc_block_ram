\begin{titlepage}
  \begin{center}

  {\Huge DC\_BLOCK\_RAM}

  \vspace{25mm}

  \includegraphics[width=0.90\textwidth,height=\textheight,keepaspectratio]{img/AFRL.png}

  \vspace{25mm}

  \today

  \vspace{15mm}

  {\Large Jay Convertino}

  \end{center}
\end{titlepage}

\tableofcontents

\newpage

\section{Usage}

\subsection{Introduction}

\par
Dual clock block RAM for any FPGA target. Includes a byte enable for selecting bytes to write from the bus.

\subsection{Dependencies}

\par
The following are the dependencies of the cores.

\begin{itemize}
  \item fusesoc 2.X
  \item iverilog (simulation)
  \item cocotb (simulation)
\end{itemize}

\input{src/fusesoc/depend_fusesoc_info.tex}

\subsection{In a Project}
\par
Connect the device using the read write signals see \ref{Module Documentation} for details

\section{Architecture}
\par
This core is made up of a single module.
\begin{itemize}
  \item \textbf{ft245\_sync\_to\_axis} Interface AXIS to F245 device (see core for documentation).
\end{itemize}

\par
This core has 2 always blocks that are sensitive to the positive clock edge.

\begin{itemize}
\item \textbf{Produce Data} Takes write input data and stores it in RAM at a specified address. BE will filter out bytes if the corresponding bits not set to active high.
\item \textbf{Consume Data} Read data from RAM at a specified address and output over read interface.
\end{itemize}

Please see \ref{Module Documentation} for information on read/write interface ports.

\section{Building}

\par
The DC block RAM is written in Verilog 2001. It should synthesize in any modern FPGA software. The core comes as a fusesoc packaged core and can be included in any other core. Be sure to make sure you have meet the dependencies listed in the previous section. Linting is performed by verible using the lint target.

\subsection{fusesoc}
\par
Fusesoc is a system for building FPGA software without relying on the internal project management of the tool. Avoiding vendor lock in to Vivado or Quartus.
These cores, when included in a project, can be easily integrated and targets created based upon the end developer needs. The core by itself is not a part of
a system and should be integrated into a fusesoc based system. Simulations are setup to use fusesoc and are a part of its targets.

\subsection{Source Files}

\subsubsection{fusesoc\_info File List}
\begin{itemize}
\item src
	\begin{itemize}
	\item src/dc\_block\_ram.v
	\end{itemize}
\item tb
	\begin{itemize}
	\item {'tb/tb\_dc\_block\_ram.v': {'file\_type': 'verilogSource'}}
	\end{itemize}
\item tb\_cocotb
	\begin{itemize}
	\item {'tb/tb\_cocotb.py': {'file\_type': 'user', 'copyto': '.'}}
	\item {'tb/tb\_cocotb.v': {'file\_type': 'verilogSource'}}
	\end{itemize}
\end{itemize}


\subsection{Targets}

\input{src/fusesoc/targets_fusesoc_info.tex}

\subsection{Directory Guide}

\par
Below highlights important folders from the root of the directory.

\begin{enumerate}
  \item \textbf{docs} Contains all documentation related to this project.
    \begin{itemize}
      \item \textbf{manual} Contains user manual and github page that are generated from the latex sources.
    \end{itemize}
  \item \textbf{src} Contains source files for the core
  \item \textbf{tb} Contains test bench files for iverilog and cocotb
\end{enumerate}

\newpage

\section{Simulation}
\par
There are a few different simulations that can be run for this core. The backend used for testing is iverilog for verilog or cocotb simulations. Usually GTKWave is used to view the fst waveform output. Cocotb are the unit tests that attempt to give a pass/fail verification to the core operation.

\subsection{iverilog}
\par
iverilog is used for simple test benches for quick visual verification of the core. This will autofinish after it has
run up to a certain number of words have been output.

\subsection{cocotb}
\par
This method allows for quick writing of test benches that actually assert and check the state of the core.
These tests are much more conclusive since it will run all test vectors and generate a report if they
pass or fail. All tests output waves to a single fst file. The method of launching the tests is to use
fusesoc. These have not been written to use a python runner method or makefiles.
To use the cocotb tests you must install the following python libraries.
\begin{lstlisting}[language=bash]
  $ pip install cocotb
\end{lstlisting}

The targets available are listed below.
\begin{itemize}
  \item \textbf{sim\_cocotb} Standard simulation for dc_block_ram.
\end{itemize}

The targets above can be run with various parameters.
This test will check the input/output against each other to validate core operation.
\begin{lstlisting}[language=bash]
  $ fusesoc run --target sim_cocotb AFRL:ram:dc_block_ram:1.0.0
\end{lstlisting}

\newpage

\section{Module Documentation} \label{Module Documentation}

\begin{itemize}
\item \textbf{dc\_block\_ram} Generic dual clock block RAM\\
\item \textbf{tb\_dc\_block\_ram-v} Verilog test bench\\
\item \textbf{tb\_cocotb-py} Cocotb python test routines\\
\item \textbf{tb\_cocotb-v} Cocotb verilog test bench\\
\end{itemize}
The next sections document the module.


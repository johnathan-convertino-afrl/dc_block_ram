\begin{titlepage}
  \begin{center}

  {\Huge DC\_BLOCK\_RAM}

  \vspace{25mm}

  \includegraphics[width=0.90\textwidth,height=\textheight,keepaspectratio]{img/AFRL.png}

  \vspace{25mm}

  \today

  \vspace{15mm}

  {\Large Jay Convertino}

  \end{center}
\end{titlepage}

\tableofcontents

\newpage

\section{Usage}

\subsection{Introduction}

\par
Dual clock block RAM for any FPGA target. Includes a byte enable for selecting bytes to write from the bus.

\subsection{Dependencies}

\par
The following are the dependencies of the cores.

\begin{itemize}
  \item fusesoc 2.X
  \item iverilog (simulation)
  \item cocotb (simulation)
\end{itemize}

\input{src/fusesoc/depend_fusesoc_info.tex}

\subsection{In a Project}
\par
Connect the device using the read write signals see \ref{Module Documentation} for details

\section{Architecture}
\par
This core is made up of a single module.
\begin{itemize}
  \item \textbf{ft245\_sync\_to\_axis} Interface AXIS to F245 device (see core for documentation).
\end{itemize}

\par
This core has 2 always blocks that are sensitive to the positive clock edge.

\begin{itemize}
\item \textbf{Produce Data} Takes write input data and stores it in RAM at a specified address. BE will filter out bytes if the corresponding bits not set to active high.
\item \textbf{Consume Data} Read data from RAM at a specified address and output over read interface.
\end{itemize}

Please see \ref{Module Documentation} for information on read/write interface ports.

\section{Building}

\par
The DC block RAM is written in Verilog 2001. It should synthesize in any modern FPGA software. The core comes as a fusesoc packaged core and can be
included in any other core. Be sure to make sure you have meet the dependencies listed in the previous section.

\subsection{fusesoc}
\par
Fusesoc is a system for building FPGA software without relying on the internal project management of the tool. Avoiding vendor lock in to Vivado or Quartus.
These cores, when included in a project, can be easily integrated and targets created based upon the end developer needs. The core by itself is not a part of
a system and should be integrated into a fusesoc based system. Simulations are setup to use fusesoc and are a part of its targets.

\subsection{Source Files}

\subsubsection{fusesoc\_info File List}
\begin{itemize}
\item src
	\begin{itemize}
	\item src/dc\_block\_ram.v
	\end{itemize}
\item tb
	\begin{itemize}
	\item {'tb/tb\_dc\_block\_ram.v': {'file\_type': 'verilogSource'}}
	\end{itemize}
\item tb\_cocotb
	\begin{itemize}
	\item {'tb/tb\_cocotb.py': {'file\_type': 'user', 'copyto': '.'}}
	\item {'tb/tb\_cocotb.v': {'file\_type': 'verilogSource'}}
	\end{itemize}
\end{itemize}


\subsection{Targets}

\input{src/fusesoc/targets_fusesoc_info.tex}

\subsection{Directory Guide}

\par
Below highlights important folders from the root of the directory.

\begin{enumerate}
  \item \textbf{docs} Contains all documentation related to this project.
    \begin{itemize}
      \item \textbf{manual} Contains user manual and github page that are generated from the latex sources.
    \end{itemize}
  \item \textbf{src} Contains source files for the core
  \item \textbf{tb} Contains test bench files for iverilog and cocotb
\end{enumerate}

\newpage

\section{Simulation}
\par
There are a few different simulations that can be run for this core.

\subsection{iverilog}
\par
iverilog is used for simple test benches for quick verification, visually, of the core.

\subsection{cocotb}
\par
This feature is not implemented for this core.

\newpage

\section{Module Documentation} \label{Module Documentation}

\begin{itemize}
\item \textbf{dc\_block\_ram} Generic dual clock block RAM\\
\end{itemize}
The next sections document the module.

